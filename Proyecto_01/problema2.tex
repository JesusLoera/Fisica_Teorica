
\renewcommand{\theenumi}{\alph{enumi}} %Letras minúsculas

\begin{enumerate}
    \item $ (AB)^{T} = B^{T} A^{T}  $
\end{enumerate}

Consideremos dos matrices cuadradas nxn, A y B:

\begin{equation*}
    A= \begin{bmatrix}
            a_{11} & a_{12} & \cdots & a_{1n} \\
            a_{21} & a_{22} & \cdots & a_{2n} \\
            \vdots  & \vdots  & \ddots  & \vdots  \\
            a_{n1} & a_{n2} & \cdots & a_{nn}
       \end{bmatrix}
    B= \begin{bmatrix}
        b_{11} & b_{12} & \cdots & b_{1n} \\
        b_{21} & b_{22} & \cdots & b_{2n} \\
        \vdots  & \vdots  & \ddots  & \vdots  \\
        b_{n1} & b_{n2} & \cdots & b_{nn}
   \end{bmatrix}
  \end{equation*}

Utilizemos la siguiente notación:

\begin{gather*}
    A= (a_{ij})_{nxn} \\
    B=(b_{ij})_{nxn}
\end{gather*}

Primero realicemos la multiplicacion de matrices $ \left( AB \right) $:

\begin{align*}
    AB   &= (a_{ij})_{nxn} (b_{ij})_{nxn} \\
                        &= \left( c_{ij} \right)_{nxn} \\
\end{align*}

La matriz resultante de dicho producto es la matriz $ \left( c_{ij} \right)_{nxn} $, donde
el término $c_{ij}$ viene dado por la siguiente expresión:

\begin{align*}
    c_{ij}    &= \Sigma_{k=1}^{n} a_{ik} b_{kj}
\end{align*}


Ahora, vamos a buscar la transpuesta de la matriz AB:

\vspace*{0.3cm}

Si $(AB)^{T} = \left( c_{ij}^{T} \right)_{nxn} $

\vspace*{0.3cm}

Por defición de matriz transpuesta, el término ij-ésimo $ c_{ij}^{T} $, 
de la matriz $(AB)^{T}$ es aquel tal que:

\begin{equation*}
    c_{ij}^{T} = c_{ji} 
\end{equation*}

Por lo tanto, tenemos que

\begin{align*}
    c_{ij}^{T}  &= c_{ji} \\
                &= \Sigma_{k=1}^{n} a_{jk} b_{ki} \\
                &= \Sigma_{k=1}^{n} b_{ki} a_{jk} 
\end{align*}

Prestemos atención en los siguientes $b_{ki}$ y $a_{jk}$.


El elemento $b_{ki}$, es el mismo que el elemento
tal que $b_{ki}=b_{ik}^{T}$. 


El elemento $a_{jk}$, es el mismo que el elemento
tal que $a_{jk}=a_{kj}^{T}$. 


Por tanto, podemos hacer la siguiente sustitución:

\begin{align*}
    c_{ij}^{T} &= \Sigma_{k=1}^{n} b_{ki} a_{jk} \\
               &= \Sigma_{k=1}^{n} b_{ik}^{T} a_{kj}^{T} 
\end{align*}

Por defición de multiplicacion de matrices el elemento 
$ c_{ij}^{T} = \Sigma_{k=1}^{n} b_{ik}^{T} a_{kj}^{T} $ es el ij-ésimo de la multiplicación de 
matrices $ \left(b_{ij}\right)_{nxn}^{T} \left(a_{ij}\right)_{nxn}^{T} $.

\vspace*{0.35cm}

Anteriormente tambíen se mostró que $ c_{ij}^{T}  $ es el ij-ésimo de la multiplicación de 
la matriz $(AB)^{T}$.

\vspace*{0.35cm}

$ \therefore $ $(AB)^{T}$ y  $ \left(b_{ij}\right)_{nxn}^{T} \left(a_{ij}\right)_{nxn}^{T} $
son matrices iguales término a término

\begin{equation*}
    \Rightarrow  (AB)^{T} = \left(b_{ij}\right)_{nxn}^{T} \left(a_{ij}\right)_{nxn}^{T}
\end{equation*}

