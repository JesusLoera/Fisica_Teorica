
\renewcommand{\theenumi}{\alph{enumi}} %Letras minúsculas

% problema a)

\textbf{ a. $ (AB)^{T} = B^{T} A^{T}  $}

\vspace*{0.5 cm}

Consideremos dos matrices cuadradas nxn, A y B:

\begin{equation*}
    A= \begin{bmatrix}
        a_{11} & a_{12} & \cdots & a_{1n} \\
        a_{21} & a_{22} & \cdots & a_{2n} \\
        \vdots & \vdots & \ddots & \vdots \\
        a_{n1} & a_{n2} & \cdots & a_{nn}
    \end{bmatrix}
    B= \begin{bmatrix}
        b_{11} & b_{12} & \cdots & b_{1n} \\
        b_{21} & b_{22} & \cdots & b_{2n} \\
        \vdots & \vdots & \ddots & \vdots \\
        b_{n1} & b_{n2} & \cdots & b_{nn}
    \end{bmatrix}
\end{equation*}

\vspace*{0.15cm}

Utilizemos la siguiente notación:

\begin{gather*}
    A= (a_{ij})_{nxn} \\
    B=(b_{ij})_{nxn}
\end{gather*}

Primero realicemos la multiplicacion de matrices $ \left( AB \right) $:

\begin{align*}
    AB & = (a_{ij})_{nxn} (b_{ij})_{nxn} \\
       & = \left( c_{ij} \right)_{nxn}   \\
\end{align*}

La matriz resultante de dicho producto es la matriz $ \left( c_{ij} \right)_{nxn} $, donde
el término $c_{ij}$ viene dado por la siguiente expresión:

\begin{align*}
    c_{ij} & = \Sigma_{k=1}^{n} a_{ik} b_{kj}
\end{align*}


Ahora, vamos a buscar la transpuesta de la matriz AB:

\vspace*{0.3cm}

Si $(AB)^{T} = \left( c_{ij}^{T} \right)_{nxn} $

\vspace*{0.3cm}

Por defición de matriz transpuesta, el término ij-ésimo $ c_{ij}^{T} $,
de la matriz $(AB)^{T}$ es aquel tal que:

\begin{equation*}
    c_{ij}^{T} = c_{ji}
\end{equation*}

Por lo tanto, tenemos que

\begin{align*}
    c_{ij}^{T} & = c_{ji}                         \\
               & = \Sigma_{k=1}^{n} a_{jk} b_{ki} \\
               & = \Sigma_{k=1}^{n} b_{ki} a_{jk}
\end{align*}

\vspace*{0.3cm}

Prestemos atención en los siguientes $b_{ki}$ y $a_{jk}$.

\vspace*{0.15cm}

El elemento $b_{ki}$, es el mismo que el elemento
tal que $b_{ki}=b_{ik}^{T}$.

\vspace*{0.15cm}

El elemento $a_{jk}$, es el mismo que el elemento
tal que $a_{jk}=a_{kj}^{T}$.

\vspace*{0.15cm}

Por tanto, podemos hacer la siguiente sustitución:

\vspace*{0.3cm}

\begin{align*}
    c_{ij}^{T} & = \Sigma_{k=1}^{n} b_{ki} a_{jk}         \\
               & = \Sigma_{k=1}^{n} b_{ik}^{T} a_{kj}^{T}
\end{align*}

Por defición de multiplicacion de matrices el elemento
$ c_{ij}^{T} = \Sigma_{k=1}^{n} b_{ik}^{T} a_{kj}^{T} $ es el ij-ésimo de la multiplicación de
matrices $ \left(b_{ij}\right)_{nxn}^{T} \left(a_{ij}\right)_{nxn}^{T} $.

\vspace*{0.5cm}

Anteriormente tambíen se mostró que $ c_{ij}^{T}  $ es el ij-ésimo de la multiplicación de
la matriz $(AB)^{T}$.

\vspace*{0.5cm}

$ \therefore $ $(AB)^{T}$ y  $ \left(b_{ij}\right)_{nxn}^{T} \left(a_{ij}\right)_{nxn}^{T} $
son matrices iguales término a término

\begin{equation*}
    \Rightarrow  (AB)^{T} = \left(b_{ij}\right)_{nxn}^{T} \left(a_{ij}\right)_{nxn}^{T}
\end{equation*}

\vspace*{1cm}

% problema b)

\noindent
\textbf{b. $ (AB)^{-1} = B^{-1} A^{-1}  $}

\vspace*{0.5cm}

\noindent
P.D: $ \left(AB\right)^{-1} = B^{-1} A^{-1} $

\vspace*{0.45cm}

\emph{Demostracion.}

\vspace{0.45cm}

Una manera de probar la veracidad de esta igualdad es reducir la expresión a una igualdad
de identidades, así demostramos que la proposición de la que partimos es verdadera.

\vspace{0.2cm}

\begin{gather*}
    \Rightarrow \left(AB\right)^{-1} = B^{-1} A^{-1}
\end{gather*}

\vspace*{0.2cm}

Multiplicamos ambos lados de la igualdad por $AB$ :

\begin{gather*}
    \Rightarrow \left(AB\right)^{-1} \left(AB\right) = B^{-1} A^{-1} \left(AB\right)
\end{gather*}

\vspace*{0.2cm}

Por defición de matriz inversa tenemos que: $ \left(AB\right)^{-1} \left(AB\right) = I $

\begin{gather*}
    \Rightarrow I =B^{-1} A^{-1} \left(AB\right)
\end{gather*}

\vspace*{0.2cm}

Por asociatividad de matrices:

\begin{align*}
    \left(B\right)^{-1} \left(A\right)^{-1} \left(AB\right) & = B^{-1} A^{-1} \left(AB\right) \\
                                                            & =B^{-1} A^{-1} AB               \\
                                                            & =B^{-1} \left(A^{-1} A\right) B
\end{align*}

\vspace*{0.2cm}

Por defición de matriz inversa: $ A^{-1} A = I $

\begin{equation*}
    \Rightarrow I = B^{-1} IB
\end{equation*}

\vspace*{0.2cm}

Como $ IB= B $

\begin{equation*}
    \Rightarrow I = B^{-1} B
\end{equation*}

\vspace*{0.2cm}

Por defición de matriz inversa: $ B^{-1} B = I $

\begin{equation*}
    \Rightarrow I = I
\end{equation*}

\vspace*{0.2cm}

Al reducir la expresión inicial a una igualdad de matrices identidad hemos demostrado la
veracidad de la proposición inicial.

\begin{equation*}
    (AB)^{-1} = B^{-1} A^{-1} \blacksquare
\end{equation*}