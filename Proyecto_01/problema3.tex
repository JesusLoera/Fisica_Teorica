Considere un proyectil que se dispara verticalmente enu un campo gravitatorio constante. Suponiendo que las velocidades
iniciales sean iguales, comparar los tiempos necesarios para que el proyectil alcance su alltura máxima:
\begin{enumerate}
    \item Cuando la fuerza resistente sea nula.
    \item Cuando la fuerza resistente es proporcional a la velocidad instantánea del proyectil.
\end{enumerate}

\noindent \textbf{Modelo matemático:} Sea $y(t)$ la altura de un proyectil en cualquier instante $t$, por la Segunda Ley 
de Newton se tiene:
\begin{align*}
    -mg-mkv &= ma \\
    -g-kv &= a \\
    -g-kv &= \frac{dv}{dt} \\
\end{align*}

\noindent \textbf{Condiciones:} $y(0)=y_{0}$, $\dot{y}(0)=v_{0}$\\
\vspace{5 mm}
\noindent \textbf{Interrogante:} $t_{1}$, tal que el proyectil alcance $y_{1,max}$ con $k=0$; $t_{2}$, tal que el proyectil
alcance $y_{2,max}$ con $k > 0$.

\vspace{5mm}
\textbf{Solución:}
\begin{align*}
    -g-kv &= \frac{dv}{dt} \\
    \frac{dv}{g+kv} &= -dt \\
    \int \frac{dv}{g+kv} &= -\int dt \\
    \frac{\ln(g+kv)}{k} &= -t + C_{1}^{*} \\
    \ln(g+kv) &= -kt + C_{1} \\
    g + kv &= e^{-kt+C_{1}} \\
    g + kv &= e^{C_{1}}e^{-kt} \\
    kv &= C_{1}e^{-kt}-g
\end{align*}

\vspace{5 mm}
Aplicando la condición inicial $v(0)=v_{0}$:
\begin{align*}
    kv_{0} &= C_{1}-g \\
    \Rightarrow C_{1} &= g+kv_{0} \\
    kv &= (g+kv_{0})e^{-kt}-g \\
    \therefore v(t) &= \frac{g+kv_{0}}{k}e^{-kt} - \frac{g}{k}
    %k\frac{dy}{dt} &= (g+kv_{0})e^{-kt}-g \\
    %kdy &= [(g+kv_{0})e^{-kt}-g]dt \\
    %\int kdy &= \int [(g+kv_{0})e^{-kt}-g]dt \\
    %ky &= -\frac{(g+kv_{0})e^{-kt}}{k} - gt + C_{2}^{*} \\
    %y &= -\frac{(g+kv_{0})e^{-kt}}{k^{2}} -\frac{g}{k}t + C_{2} \\
\end{align*}

%\vspace{5 mm}
%Aplicando la condición inicial $y(0)=y_{0}$:
%\begin{align*}
%    y_{0} &= -\frac{g+kv_{0}}{k^{2}} + C_{2} \\
%    \Rightarrow C_{2} &= y_{0} + \frac{g+kv_{0}}{k^{2}} \\
%    y &= -\frac{(g+kv_{0})e^{-kt}}{k^{2}} -\frac{g}{k}t + y_{0} + \frac{g+kv_{0}}{k^{2}} \\
%    \therefore y(t) &= \frac{g+kv_{0}}{k^{2}} (1-e^{-kt}) - \frac{g}{k}t + y_{0}
%\end{align*}

%\vspace{5 mm}
%Se tiene: $kv = C_{1}e^{-kt}-g$. Se busca $t^{*}$, tal que $v(t^{*})=0$.
%\begin{align*}
%    0 &= (g+kv_{0})e^{-kt^{*}}-g \\
%    e^{-kt^{*}} &= \frac{g}{g+kv_{0}} \\
%    -kt^{*} &= \ln \left(\frac{g}{g+kv_{0}}\right) \\
%    \Rightarrow t^{*} &= -\frac{1}{k} \ln \left(\frac{g}{g+kv_{0}}\right)
%\end{align*}

%Sustituyendo $t^{*}$ en $y(t)$:
%\begin{equation*}
%    y(t^{*}) = \frac{g+kv_{0}}{k^{2}} \left(1-\frac{g}{g+kv_{0}}\right) + \frac{g}{k^{2}} \ln \left(\frac{g}{g+kv_{0}} \right) + y_{0}
%\end{equation*}

\vspace{5 mm}
Para hallar el tiempo necesario para que el proyectil alcance $y_{1,max}$ (en un medio con $k>0$), se busca un $t_{1}$ tal que $v(t_{1})=0$.
\begin{align*}
    0 &= (g+kv_{0})e^{-kt_{1}}-g \\
    e^{-kt_{1}} &= \frac{g}{g+kv_{0}} \\
    -kt_{1} &= \ln \left(\frac{g}{g+kv_{0}}\right) \\
    \Rightarrow t_{1} &= -\frac{1}{k} \ln \left(\frac{g}{g+kv_{0}}\right)
\end{align*}

\vspace{5 mm}
En un medio sin resistencia al aire, es decir, con $k=0$, se puede hallar el tiempo para que el proyectil alcance 
$y_{2,max}$ a partir de las ecuaciones básicas de cinemática.
\begin{align*}
    0 &= v_{0} -gt_{2} \\
    v_{0} &= gt_{2} \\
    \Rightarrow t_{2} &= \frac{v_{0}}{g}
\end{align*}

\vspace{5 mm}
A manera de comparación, se obtiene la razón entre el tiempo que le toma al proyectil con resistencia al aire alcanzar
su altura máxima y el tiempo que le toma al proyectil en un medio sisn resistencia. Esto es:
\begin{align*}
    \frac{t_{1}}{t_{2}} &= \frac{-\frac{1}{k} \ln \left(\frac{g}{g+kv_{0}}\right)}{\frac{v_{0}}{g}} \\
    \therefore t_{1} &= \frac{-\frac{1}{k} \ln \left(\frac{g}{g+kv_{0}}\right)}{\frac{v_{0}}{g}} t_{2}
\end{align*}