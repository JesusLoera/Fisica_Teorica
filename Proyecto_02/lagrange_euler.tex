
% PÁGINA 1

\noindent

Solución.

\vspace*{0.5cm}

Consideremos la integral:

\begin{equation}
    J = \int_{x_{1}}^{x_{2}} f\left[ y, z, y_{x}, z_{x}; x \right] dx
\end{equation}

Donde tenemos:

\begin{gather}
    y_{x} = \frac{\partial y}{\partial x} \\
    z_{x} = \frac{\partial z}{\partial x}
\end{gather}

\vspace*{0.5cm}

Y también una restricción de la forma:

\vspace*{0.5cm}

\begin{equation}
    g\left[ y,z;x \right] = 0
\end{equation}

\vspace*{0.5cm}

Consideremos la condición para que J sea un valor extremo:

\vspace*{0.5cm}

\begin{equation}
    \left[ \frac{\partial J}{\partial \alpha} \right]_{\alpha = 0} = 0
\end{equation}

\vspace*{0.5cm}

Primeramente definamos funciones axiliares de y,z, en términos del
parámetro de variación $\alpha$ y la variable independiente $x$:

\begin{gather}
    y(\alpha,x) = y(0,x) + \alpha \eta_{1} (x) \\
    z(\alpha,x) = z(0,x) + \alpha \eta_{2} (x)
\end{gather}

\vspace*{0.5cm}

Con las siguientes condiciones:

\begin{enumerate}
    \item En los extremos $\eta_{1} (x)$ y $\eta_{2} (x)$ son iguales a cero.
    \item $\eta_{1} (x)$ y $\eta_{2} (x)$ son diferenciables en $(x_{1}, x_{2})$
\end{enumerate}

% PÁGINA 2

\vspace{0.5cm}

Aplicamos la condición (5) a (1):

\begin{equation*}
    \frac{\partial J}{\partial \alpha} =
    \frac{\partial}{\partial \alpha}
    \left[
        \int_{x_{1}}^{x_{2}}
        f\left[ y, z, y_{x}, z_{x}; x \right]
        dx
        \right]
\end{equation*}

\vspace*{0.5cm}

Como en el lado derecho de la ecuación los límites de la integral están fijos, es posible aplicar la derivada parcial
con respecto a alfa al integrando mediante la regla de la cadena:

\vspace*{0.5cm}

\begin{equation}
    \frac{\partial J}{\partial \alpha} =
    \int_{x_{1}}^{x_{2}}
    \left[
        \frac{\partial f}{\partial y} \frac{\partial y}{\partial \alpha} +
        \frac{\partial f}{\partial y_{x}} \frac{\partial y_{x}}{\partial \alpha} +
        \frac{\partial f}{\partial z} \frac{\partial z}{\partial \alpha} +
        \frac{\partial f}{\partial z_{x}} \frac{\partial z_{x}}{\partial \alpha}
        \right]
    dx
\end{equation}

\vspace*{0.5cm}

Derivando parcialmente respecto a $\alpha$ las ecuaciones de (6) y (7)

\begin{align*}
    \frac{\partial}{\partial \alpha} y(\alpha,x) & = \frac{\partial}{\partial \alpha} \left[ y(0,x) + \alpha \eta_{1} (x) \right]                    \\
                                                 & = \frac{\partial}{\partial \alpha}  y(0,x) + \frac{\partial}{\partial \alpha} \alpha \eta_{1} (x) \\
                                                 & = \eta_{1} (x)
\end{align*}

\vspace*{0.5cm}

\begin{align*}
    \frac{\partial}{\partial \alpha} z(\alpha,x) & = \frac{\partial}{\partial \alpha} \left[ z(0,x) + \alpha \eta_{2} (x) \right]                    \\
                                                 & = \frac{\partial}{\partial \alpha}  z(0,x) + \frac{\partial}{\partial \alpha} \alpha \eta_{2} (x) \\
                                                 & = \eta_{2} (x)
\end{align*}

\vspace*{0.5cm}

Por tanto, tenemos las siguientes ecuaciones:

\begin{gather}
    \frac{\partial y}{\partial \alpha} = \eta_{1} (x) \\
    \frac{\partial z}{\partial \alpha} = \eta_{2} (x)
\end{gather}

\vspace*{0.5cm}

Ahora hallamos $\frac{\partial y_{x}}{\partial \alpha}$ y $\frac{\partial z_{x}}{\partial \alpha}$ :

\vspace*{0.5cm}

\begin{align*}
    \frac{\partial y_{x}}{\partial \alpha} & =  \frac{\partial}{\partial \alpha} \frac{dy}{dx} \\
                                           & = \frac{\partial}{\partial \alpha} \frac{d}{dx}
    \left[ y(0,x) + \alpha \eta_{1} (x) \right]                                                \\
                                           & = \frac{\partial}{\partial \alpha}
    \left[ y^{'} (0,x) + \alpha \eta_{1}^{'} (x) \right]                                       \\
                                           & = \eta_{1}^{'} (x)                                \\
                                           & = \frac{d \eta_{1}}{dx}
\end{align*}

\vspace*{0.5cm}

\begin{align*}
    \frac{\partial z_{x}}{\partial \alpha} & = \frac{\partial}{\partial \alpha} \frac{dz}{dx} \\
                                           & = \frac{\partial}{\partial \alpha} \frac{d}{dx}
    \left[ z(0,x) + \alpha \eta_{2} (x) \right]                                               \\
                                           & = \frac{\partial}{\partial \alpha}
    \left[ z^{'} (0,x) + \alpha \eta_{2}^{'} (x) \right]                                      \\
                                           & = \eta_{2}^{'} (x)                               \\
                                           & = \frac{d \eta_{2}}{dx}
\end{align*}

\vspace*{0.5cm}

% PÁGINA 3

Por tanto tenemos las siguientes ecuaciones:

\begin{gather}
    \frac{\partial y_{x}}{\partial \alpha} = \frac{d \eta_{1}}{dx} \\
    \frac{\partial z_{x}}{\partial \alpha} = \frac{d \eta_{2}}{dx}
\end{gather}

\vspace*{0.5cm}

Sustituyendo (9), (10), (11) y (12) en (8):

\vspace*{0.5cm}

\begin{equation*}
    \frac{\partial J}{\partial \alpha} =
    \int_{x_{1}}^{x_{2}}
    \left[
        \frac{\partial f}{\partial y} \eta_{1}  +
        \frac{\partial f}{\partial y_{x}}  \frac{d \eta_{1}}{dx} +
        \frac{\partial f}{\partial z} \eta_{2}  +
        \frac{\partial f}{\partial z_{x}}  \frac{d \eta_{2}}{dx}
        \right]
    dx
\end{equation*}


Separamos la integral en sumas de integrales:

\vspace*{0.5cm}

\begin{equation}
    \frac{\partial J}{\partial \alpha} =
    \int_{x_{1}}^{x_{2}} \frac{\partial f}{\partial y} \eta_{1}   dx
    +\int_{x_{1}}^{x_{2}} \frac{\partial f}{\partial y_{x}}  \frac{d \eta_{1}}{dx} dx
    +\int_{x_{1}}^{x_{2}} \frac{\partial f}{\partial z} \eta_{2}  dx
    +\int_{x_{1}}^{x_{2}} \frac{\partial f}{\partial z_{x}}  \frac{d \eta_{2}}{dx} dx
\end{equation}

Prestemos atención en los siguientes términos

$$
    \int_{x_{1}}^{x_{2}} \frac{\partial f}{\partial y_{x}}  \frac{d \eta_{1}}{dx} dx
$$

$$
    \int_{x_{1}}^{x_{2}} \frac{\partial f}{\partial z_{x}}  \frac{d \eta_{2}}{dx} dx
$$

de la ecuación (13).

\vspace*{0.5cm}

\begin{itemize}
    \item $$\int_{x_{1}}^{x_{2}} \frac{\partial f}{\partial y_{x}}  \frac{d \eta_{1}}{dx} dx$$
\end{itemize}

Hacemos $u=\frac{\partial f}{\partial y_{x}}$ y $dv=\frac{d \eta_{1}}{dx} dx$

\vspace*{0.5cm}

Entonces $du=\frac{d}{dx} \frac{\partial f}{\partial y_{x}} dx$ y $v=\eta_{1} (x)$

\vspace*{0.5cm}

Por lo tanto

\begin{equation*}
    \int_{x_{1}}^{x_{2}} \frac{\partial f}{\partial y_{x}}  \frac{d \eta_{1}}{dx} dx =
    \left[ \frac{\partial f}{\partial y_{x}} \eta_{1} (x) \right]  \bigg\rvert_{x_{1}}^{x_{2}} -
    \int_{x_{1}}^{x_{2}} \eta_{1} (x) \frac{d}{dx} \frac{\partial f}{\partial y_{x}} dx
\end{equation*}

\vspace*{0.5cm}

Como $\eta_{1} (x_{1})=\eta_{1} (x_{2})=0$ implica que:

\begin{equation*}
    \left[ \frac{\partial f}{\partial y_{x}} \eta_{1} (x) \right]  \bigg\rvert_{x_{1}}^{x_{2}} = 0
\end{equation*}

\vspace*{0.5cm}

Sustituyendo esta expresión en la integral anterior:

\vspace*{0.5cm}

\begin{equation}
    \int_{x_{1}}^{x_{2}} \frac{\partial f}{\partial y_{x}}  \frac{d \eta_{1}}{dx} dx =
    -\int_{x_{1}}^{x_{2}} \eta_{1} (x) \frac{d}{dx} \frac{\partial f}{\partial y_{x}} dx
\end{equation}