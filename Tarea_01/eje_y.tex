Para el desplazamiento vertical se procede de la siguiente forma:

Como $\ddot{y}=\frac{dV_{y}}{dt}$ y $\dot{V_{y}}$:

\begin{gather*}
    m\frac{dV_{y}}{dt} = -kmV_{y}-mg\\
    -\frac{dV_{y}}{dt} = kV_{y}+g\\
    \int \frac{dV_{y}}{kV_{y}+g} = -\int dt
\end{gather*}

Resolviendo la integral de la izquierda:
Sea $u=kV_{y}+g$, $du=kdV_{y}$
\begin{equation*}
    \Rightarrow \frac{1}{k} \int \frac{du}{u} = \frac{1}{k} \ln(\lvert u \rvert) = \frac{1}{k} \ln(\lvert kV_{y}+g \rvert)
\end{equation*}

Asumiendo $kV_{y}+g \geq 0$:
\begin{gather*}
    \Rightarrow \frac{1}{k} \ln(kV_{y}+g) = -t+C_{1}\\
    \frac{1}{k} \ln(kV_{y}+g) = -kt+C_{1} \hspace{3cm} (C_{1}=kC_{1})\\
    kV_{y}+g = e^{-kt}e^{C_{1}}\\
    kV_{y}+g = C_{1}e^{-kt} \hspace{3cm} (C_{1}=e^{C_{1}})
\end{gather*}