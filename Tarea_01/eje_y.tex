Para el desplazamiento vertical se procede de la siguiente forma:
\vspace{5 mm}

Como $\ddot{y}=\frac{dV_{y}}{dt}$ y $\dot{y}=V_{y}$:
\begin{align*}
    m\frac{dV_{y}}{dt} &= -kmV_{y}-mg\\
    -\frac{dV_{y}}{dt} &= kV_{y}+g\\
    \int \frac{dV_{y}}{kV_{y}+g} &= -\int dt
\end{align*}

\vspace{5 mm}
Resolviendo la integral de la izquierda:
Sea $u=kV_{y}+g$, $du=kdV_{y}$
\begin{equation*}
    \Rightarrow \frac{1}{k} \int \frac{du}{u} = \frac{1}{k} \ln(\lvert u \rvert) = \frac{1}{k} \ln(\lvert kV_{y}+g \rvert)
\end{equation*}

\vspace{5 mm}
Asumiendo $kV_{y}+g \geq 0$:
\begin{align*}
    \Rightarrow \frac{1}{k} \ln(kV_{y}+g) &= -t+C_{1}\\
    \frac{1}{k} \ln(kV_{y}+g) &= -kt+C_{1} \hspace{3cm} (C_{1}=kC_{1})\\
    kV_{y}+g &= e^{-kt}e^{C_{1}}\\
    kV_{y}+g &= C_{1}e^{-kt} \hspace{3cm} (C_{1}=e^{C_{1}})
\end{align*}

\vspace{5 mm}
Aplicando la condición incial $V_{y}(t=0)=V_{0}\sin\theta$:
\begin{align*}
    k(V_{0}\sin\theta)+g &= C_{1}e^{-k(0)}\\
    \Rightarrow C_{1} &= kV_{0}\sin\theta+g
\end{align*}

\begin{align*}
    \Rightarrow kV_{y}+g &= (kV_{0}\sin\theta+g)e^{-kt}\\
    V_{y} &= \frac{(kV_{0}\sin\theta+g)e^{-kt}-g}{k}\\
    V_{y} &= (V_{0}\sin\theta)e^{-kt} + \frac{g}{k}e^{-kt} - \frac{g}{k}\\
    V_{y}(t) &= \left (\frac{kV_{0}\sin\theta+g}{k} \right)e^{-kt} - \frac{g}{k}
\end{align*}