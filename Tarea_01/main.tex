\documentclass{article}
\usepackage{blindtext}
\usepackage[T1]{fontenc}
\usepackage[utf8]{inputenc}
\usepackage{amsmath}
\usepackage{amsfonts}
\usepackage{color}
\usepackage{graphicx}
\usepackage{vmargin}
\usepackage{amssymb}

\setlength{\jot}{10pt}
\setmargins{2.5cm}       % margen izquierdo
{1.5cm}                        % margen superior
{16.5cm}                      % anchura del texto
{23.42cm}                    % altura del texto
{10pt}                           % altura de los encabezados
{1cm}                           % espacio entre el texto y los encabezados
{0pt}                             % altura del pie de página
{2cm}                           % espacio entre el texto y el pie de página


\renewcommand{\abstractname}{Descripción}   % Cambiamos el nombre del abstract

\makeatletter
\newcommand*{\bigcdot}{}% Check if undefined
\DeclareRobustCommand*{\bigcdot}{%
  \mathbin{\mathpalette\bigcdot@{}}%
}
\newcommand*{\bigcdot@scalefactor}{.5}
\newcommand*{\bigcdot@widthfactor}{1.15}
\newcommand*{\bigcdot@}[2]{%
  % #1: math style
  % #2: unused
  \sbox0{$#1\vcenter{}$}% math axis
  \sbox2{$#1\cdot\m@th$}%
  \hbox to \bigcdot@widthfactor\wd2{%
    \hfil
    \raise\ht0\hbox{%
      \scalebox{\bigcdot@scalefactor}{%
        \lower\ht0\hbox{$#1\bullet\m@th$}%
      }%
    }%
    \hfil
  }%
}
\makeatother

\title{Tarea 1}
\author{Jesus Eduardo Loera Casas 1898887,\\ Cesar Efrén Valladares Rocha 1841555,\\ Vrani Chavez Islas 1990044}
\date{\today}

\begin{document}

\begin{titlepage}
    \centering
    {\includegraphics[width=0.2\textwidth]{FCFM.png}\par}
    \vspace{1cm}
    {\bfseries\LARGE Universidad Autonoma de Nuevo León \par}
    \vspace{1cm}
    {\scshape\Large Facultad de Ciencias Físico Matemáticas \par}
    \vspace{3cm}
    {\scshape\Huge Proyecto 2  \par}
    \vspace{3cm}
    {\Large Autores: \par}
    \vfill
    {\Large Jesús Eduardo Loera Casas 1898887 \par}
    \vfill{\Large Cesar Efrén Valladares Rocha 1841555 \par}
    \vfill
    {\Large Vrani Chavez Islas 1990044 \par}
    \vfill
    {\Large \today \par}
\end{titlepage}

\tableofcontents			% Índice

\begin{center}
	\rule[0mm]{150mm}{0.1mm}		% Para dibujar una linea horizontal de
									% [elevación]{longitud}{grosor}
	\end{center}
	
	
\begin{abstract}		% ABSTRACT

	\noindent 				% Anula la sangria de este parrafo
	En este documento nuestro equipo presenta la tarea 1 del curso de mecánica teórica, donde planteamos 
  el problema de una particula moviendose en un medio resistente y encontramos sus ecuaciones de 
  movimiento.
	\end{abstract}
	
\begin{center}
	\rule[0mm]{150mm}{0.1mm}
	\end{center}

  \section{Problema}		
                            
  Encontrar las ecuaciones de para la velocidad respecto al tiempo y el movimiento respecto
  al tiempo de una partícula que se mueve en un medio resistente en una trayectoria parabólica 
  con las siguientes condiciones iniciales:

  \begin{gather*}
    x(t=0)=0=y(t=0) \\
    \dot{x} (t=0) = V_{0} Cos\theta \\
    \dot{y} (t=0) = V_{0} Sen\theta 
  \end{gather*}

  Las ecuaciones de movimiento que describen la trayectoria del sistema son:

  \begin{align*}
    m\ddot{x} &= -km\dot{x} \\
    m\ddot{y} &= -km\dot{y}-mg
  \end{align*}

  Hallar: $ x(t), \dot{x} (t) , y(t), \dot{y} (t) $

\section{Solución}

 Observamos un sistema desacoplado de dos ecuaciones diferenciales.

\vspace*{5 mm}

Empezaremos con la ecuación diferencial: $ m\ddot{x} = -km\dot{x} $

\vspace*{5 mm}

Por comodidad usaremos momentaneamente la siguiente notación: 

\begin{gather*}
    \dot{x} = V_{x} \\
    \ddot{x} = \frac{dV_{x}}{dt}
\end{gather*}

Empezamos realizando una sustitución y simplificamos

\begin{gather*}
    m\frac{d V_{x}}{dt} = -km\dot{x} \\
    \frac{d V_{x}}{dt} = -k\dot{x} \\
    \frac{ d V_{x}}{V_{x}} = -kdt 
\end{gather*}

Integramos  ambos lados de la ecuación diferencial

\begin{gather*}
    \int \frac{ d V_{x}}{V_{x}} = \int -kdt \\
    ln(V_{x}) = -kt+C_{1}
\end{gather*}

Simplificando la expresión

\begin{gather*}
    e^{ln(V_{x})} = e^{-kt+C_{1}} \\
    V_{x} = e^{-kt} e^{C_{1}} \\
    V_{x} = C_{1}^{*}e^{-kt}  \\
    V_{x} (t) = C_{1}^{*}e^{-kt}
\end{gather*}

Evaluamos la condición inicial:

\begin{itemize}
    \item $\dot{x} (0) = V_{x} (0) = V_{0}cos \theta$
\end{itemize}

\begin{gather*}
    V_{0}cos \theta =  C_{1}^{*}e^{-k(0)} \\
    \Longrightarrow C_{1}^{*} = V_{0}cos \theta
\end{gather*}

Sustituyendo $C_{1}^{*} = V_{0}cos \theta$ en $V_{x} (t)$

\begin{equation*}
    V_{x} (t) = V_{0}cos \theta e^{-kt}
\end{equation*}


 Para el desplazamiento vertical se procede de la siguiente forma:

Como $\ddot{y}=\frac{dV_{y}}{dt}$ y $\dot{V_{y}}$:

\begin{gather*}
    m\frac{dV_{y}}{dt} = -kmV_{y}-mg\\
    -\frac{dV_{y}}{dt} = kV_{y}+g\\
    \int \frac{dV_{y}}{kV_{y}+g} = -\int dt
\end{gather*}

Resolviendo la integral de la izquierda:
Sea $u=kV_{y}+g$, $du=kdV_{y}$
\begin{equation*}
    \Rightarrow \frac{1}{k} \int \frac{du}{u} = \frac{1}{k} \ln(\lvert u \rvert) = \frac{1}{k} \ln(\lvert kV_{y}+g \rvert)
\end{equation*}

Asumiendo $kV_{y}+g \geq 0$:
\begin{gather*}
    \Rightarrow \frac{1}{k} \ln(kV_{y}+g) = -t+C_{1}\\
    \frac{1}{k} \ln(kV_{y}+g) = -kt+C_{1} \hspace{3cm} (C_{1}=kC_{1})\\
    kV_{y}+g = e^{-kt}e^{C_{1}}\\
    kV_{y}+g = C_{1}e^{-kt} \hspace{3cm} (C_{1}=e^{C_{1}})
\end{gather*}

\end{document}