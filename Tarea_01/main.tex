\documentclass{article}
\usepackage{blindtext}
\usepackage[T1]{fontenc}
\usepackage[utf8]{inputenc}
\usepackage{amsmath}
\usepackage{amsfonts}
\usepackage{color}
\usepackage{graphicx}
\usepackage{vmargin}

\setmargins{2.5cm}       % margen izquierdo
{1.5cm}                        % margen superior
{16.5cm}                      % anchura del texto
{23.42cm}                    % altura del texto
{10pt}                           % altura de los encabezados
{1cm}                           % espacio entre el texto y los encabezados
{0pt}                             % altura del pie de página
{2cm}                           % espacio entre el texto y el pie de página


\renewcommand{\abstractname}{Descripción}   % Cambiamos el nombre del abstract

\makeatletter
\newcommand*{\bigcdot}{}% Check if undefined
\DeclareRobustCommand*{\bigcdot}{%
  \mathbin{\mathpalette\bigcdot@{}}%
}
\newcommand*{\bigcdot@scalefactor}{.5}
\newcommand*{\bigcdot@widthfactor}{1.15}
\newcommand*{\bigcdot@}[2]{%
  % #1: math style
  % #2: unused
  \sbox0{$#1\vcenter{}$}% math axis
  \sbox2{$#1\cdot\m@th$}%
  \hbox to \bigcdot@widthfactor\wd2{%
    \hfil
    \raise\ht0\hbox{%
      \scalebox{\bigcdot@scalefactor}{%
        \lower\ht0\hbox{$#1\bullet\m@th$}%
      }%
    }%
    \hfil
  }%
}
\makeatother

% \title{Tarea 1}
% \author{Jesus Eduardo Loera Casas 1898887,\\ Cesar Efrén Valladares Rocha 1841555,\\ Vrani Chavez Islas 1990044}
% \date{\today}

\begin{document}

<<<<<<< HEAD
\begin{titlepage}
    \centering
    {\includegraphics[width=0.2\textwidth]{FCFM.png}\par}
    \vspace{1cm}
    {\bfseries\LARGE Universidad Autonoma de Nuevo León \par}
    \vspace{1cm}
    {\scshape\Large Facultad de Ciencias Físico Matemáticas \par}
    \vspace{3cm}
    {\scshape\Huge Proyecto 2  \par}
    \vspace{3cm}
    {\Large Autores: \par}
    \vfill
    {\Large Jesús Eduardo Loera Casas 1898887 \par}
    \vfill{\Large Cesar Efrén Valladares Rocha 1841555 \par}
    \vfill
    {\Large Vrani Chavez Islas 1990044 \par}
    \vfill
    {\Large \today \par}
\end{titlepage}
=======
\input{problema_1.tex}

\maketitle      			% Hoja con el titulo, autor y fecha
>>>>>>> 0132e06276dc817d2994f808c56502fe233e634e

\tableofcontents			% Índice

\begin{center}
	\rule[0mm]{150mm}{0.1mm}		% Para dibujar una linea horizontal de
									% [elevación]{longitud}{grosor}
	\end{center}
	
	
\begin{abstract}		% ABSTRACT

	\noindent 				% Anula la sangria de este parrafo
	En ésta tarea demostraremos algunas de las propiedades de los vectores en 
	$\mathbb{R}^{n}$
	\end{abstract}
	
\begin{center}
	\rule[0mm]{150mm}{0.1mm}
	\end{center}

\section{Introduction}		% Introduce el primer tema, pero el asterisco le 
                            % quita la enumeración
                            
    Este es un resumen con los apuntes más importantes de la tercera sesión del curso de mecánica teórica

\section{Problema}

\section{Solución}




	
\end{document}