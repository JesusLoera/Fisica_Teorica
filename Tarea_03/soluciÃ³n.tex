

 Solución.

 \begin{align*}
    L\ddot{q} + \frac{1}{C} q &= 0 
\end{align*}

Observación: $L\ddot{q} + \frac{1}{C} q = 0 $ es una ecuación diferencial lineal
homogénea de orden dos con coeficientes constantes.


Observación: Tenemos condiciones iniciales en $t=0$ por tanto podemos
usar la transformada de Laplace para resolver la ecuación diferencial.

\vspace*{0.3 cm}

Primeramente aplicamos la transformada de Laplace de ambos lados de la ecuación:

\begin{align*}
    \Laplace \left\{ L\ddot{q} + \frac{1}{C} q \right\} &= \Laplace \left\{ 0 \right\}
\end{align*}

Como $ \Laplace \left\{ 0 \right\} = 0 $

\begin{align*}
    \Laplace \left\{ L\ddot{q} \right\} + \Laplace \left\{\frac{1}{C} q \right\} &= 0 \\
    L \Laplace \left\{ \ddot{q} \right\} + \frac{1}{C} \Laplace \left\{q  \right\} &= 0
\end{align*}

Sea $\Laplace \{ q(t) \} = Q(s)$ y recordando que $ q(0)= q_{o} $ y $I(0) = \dot{q} (0)=0$ 

\begin{itemize}
    \item En $L \Laplace \left\{ \ddot{q} \right\} $: 
\end{itemize}

\begin{align*}
    L \Laplace \left\{ \ddot{q} \right\} &= L \left( s^{2}Q(s) -sq(0) - \dot{q}(0) \right) \\
                                         &= L \left( s^{2}Q(s) -sq_{o} - 0 \right) \\
                                         &= L s^{2}Q(s) -Lsq_{o}
\end{align*}

\begin{itemize}
    \item En $\frac{1}{C} \Laplace \left\{ q \right\} $: 
\end{itemize}

\begin{align*}
    \frac{1}{C} \Laplace \left\{ q \right\} &= \frac{1}{C} Q(s)
\end{align*}

Sustituyemos las anteriores igualdades en la ecuación diferencial:

\begin{align*}
    L s^{2}Q(s) -Lsq_{o} + \frac{1}{C} Q(s) = 0 
\end{align*}

Ahora despejamos para $Q(s)$ :

\begin{align*}
    L s^{2}Q(s) -Lq_{o}s + \frac{1}{C} Q(s) = 0 \\
    L s^{2}Q(s) + \frac{1}{C} Q(s) = Lq_{o}s 
\end{align*}

\begin{align*}
    Q(s) \left( L s^{2} + \frac{1}{C} \right) = Lq_{o}s \\
    Q(s) \left( \frac{CL s^{2}+1}{C} \right) = Lq_{o}s \\
    Q(s)  = \frac{CLq_{o}s}{CL s^{2}+1} \\
    Q(s)  = \frac{CLq_{o}s}{ CL \left( s^{2} + \frac{1}{CL} \right) } \\
    Q(s)  = \frac{q_{o}s}{  s^{2} + \frac{1}{CL}  }
\end{align*}

Ahora aplicamos la transformada inversa de Laplace en ambos lados de la ecuación:

%   \Laplace^{-1} \{  \}

\begin{align*}
    \Laplace^{-1} \{ Q(s) \} &= \Laplace^{-1} \left\{\frac{q_{o}s}{  s^{2} + \frac{1}{CL}  }\right\} \\
                            &= q_{o} \Laplace^{1} \left\{\frac{s}{ s^{2} + \frac{1}{CL} }\right\} 
\end{align*}

Como  $ \Laplace^{-1} \left\{\frac{s}{s^{2}+k^{2}}\right\} = Cos(kt) $

\begin{equation*}
    \Rightarrow \Laplace \left\{ \frac{s}{ s^{2} + \frac{1}{CL}  } \right\} = Cos \left( \sqrt{ \frac{1}{CL} } t \right)
\end{equation*}

Tambien sabemos que $ \Laplace^{-1} \left\{Q(s)\right\} = q(t) $


Hacemos las sustituciones de las transformadas inversas de Laplace:

\begin{equation*}
    \therefore q(t) = q_{o} Cos \left( \sqrt{ \frac{1}{CL} } t \right)
\end{equation*}

Ahora para determinar $\dot{q} (t)$ derivamos a $q(t)$ :

\begin{align*}
    q(t) &= q_{o} Cos \left( \sqrt{ \frac{1}{CL} } t \right) \\
    \therefore \dot{q}(t) &= - \sqrt{ \frac{1}{CL} } q_{o} Sen \left(  \sqrt{ \frac{1}{CL} } t \right)
\end{align*}