


\subsection{Forma 1}

    Considere la identidad: $ Cos(\alpha - \beta) = Cos(\alpha)Cos(\beta)+Sen(\alpha)Sen(\beta) $

    \vspace*{0.3cm}

    Ahora retomamos la solución general $x(t) = C_{1}Cos(\omega_{0}t) + C_{2}Sen(\omega_{0}t)$ y proponemos los
    siguientes igualdades: $C_{1}=Acos(\phi)$, $C_{2}=ASen(\phi)$

    \begin{align*}
        x(t) &= C_{1}Cos(\omega_{0}t) + C_{2}Sen(\omega_{0}t) \\
        x(t) &= ACos(\phi)Cos(\omega_{0}t) + ASen(\phi)Sen(\omega_{0}t) \\
        x(t) &= A \left( Cos(\phi)Cos(\omega_{0}t) + Sen(\phi)Sen(\omega_{0}t) \right) \\
        x(t) &= A \left( Cos(\omega_{0}t)Cos(\phi) + Sen(\omega_{0}t)Sen(\phi) \right) 
    \end{align*}

    Observamos la misma forma de la identidad trigonométrica planteada el inicio, entonces:

    \begin{equation*}
        x(t) = ACos(\omega_{0}t-\phi)
    \end{equation*}


\subsection{Forma 2}

    Se suponen las condiciones iniciales $x(t=0)=A$, $\dot{x}(t=0)=0$.\\
    Aplicando la condición $x(t=0)=A$:
    \begin{align*}
        A &= C_{1}Cos(0)+C_{2}Sen(0)\\
        \Rightarrow C_{1} &= A
    \end{align*}

    Sustituyendo $C_{1}=A$ en $x(t)$:
    \begin{align*}
        x(t) &= ACos(\omega_{0}t)+C_{2}Sen(\omega_{0}t)\\
        \Rightarrow \dot{x}(t) &= -A\omega_{0}Sen(\omega_{0}t) + C_{2}\omega_{0}Cos(\omega_{0}t)
    \end{align*}

    Aplicando la condición inicial $\dot{x}(t=0)=0$:
    \begin{align*}
        0 &= -A\omega_{0}Sen(0) + C_{2}\omega_{0}Cos(0)\\
        0 &= C_{2}\omega_{0}\\
        \Rightarrow C_{2} &= 0
    \end{align*}

    Sustituyendo $C_{2}=0$ en $x(t)$:
    \begin{equation*}
        x(t) = ACos(\omega_{0}t)
    \end{equation*}

    Se agrega una constante de desfase $\phi$ a la función $x(t)$ como desplazamiento dentro del ángulo del coseno. Esto con el fin de que la función sea compatible con cualquier sistema oscilatorio con un desfase inicial respecto al punto de equilibrio.
    \begin{equation*}
        \therefore x(t) = ACos(\omega_{0}t-\phi)
    \end{equation*}

    Notemos que dadas las condiciones iniciales, en este caso particular $\phi=0$.\\
    Si se hace uso de la identidad $Sen(x+\frac{\pi}{2}) = Cos(x)$ se puede reescribir $x(t)$ como:
    \begin{align*}
        x(t) &= ASen((\omega_{0}t-\phi)+\frac{\pi}{2})\\
        x(t) &= ASen(\omega_{0}t-(\phi-\frac{\pi}{2}))\\
        \therefore x(t) &= ASen (\omega_{0}t-\delta)
    \end{align*}
    donde $\delta=\phi - \frac{\pi}{2}$