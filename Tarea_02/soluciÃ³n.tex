

 Solución.

 \begin{align*}
    \ddot{x} + \omega_{0}^{2} x &= 0 \\
    \frac{d^{2} x}{dt^{2}} + \omega_{0}^{2} x &= 0 
\end{align*}

Observación: $\frac{d^{2} x}{dt^{2}} + \omega_{0}^{2} x = 0 $ es una ecuación diferencial lineal
no homogénea de orden dos con coeficientes constantes.

\vspace*{0.3 cm}

Identificamos la ecuación característica: $ m^{2} + \omega_{0}^{2} = 0 $

Ahora buscamos despejar para m:

\begin{align*}
    m^{2} + \omega_{0}^{2} &= 0 \\
    m^{2}  &= - \omega_{0}^{2} \\
    \Rightarrow m &= 0 \pm \omega_{0} i
\end{align*}

Observación: Hemos obtenido raíces complejas conjugadas.

\vspace*{0.3 cm}

Identificamos $ a=0 y b=\omega $, la solución general de la ecuación diferencial
es de la forma:

\begin{align*}
    x_{g}(t) &= C_{1}e^{at}Cos(bt) + C_{2}e^{at}Sen(bt) \\
\end{align*}

Sustituyendo a y b:

\begin{align*}
    x_{g}(t) &= C_{1}e^{(0)t}Cos((\omega_{0})t) + C_{2}e^{(0)t}Sen((\omega_{0})t) \\
    x_{g}(t) &= C_{1}(1)Cos(\omega_{0}t) + C_{2}(1)Sen(\omega_{0}t) \\
    \Rightarrow x_{g}(t) &= C_{1}Cos(\omega_{0}t) + C_{2}Sen(\omega_{0}t)
\end{align*}

Se suponen las condiciones iniciales $x(t=0)=A$, $\dot{x}(t=0)=0$.\\
Aplicando la condición $x(t=0)=A$:
\begin{align*}
    A &= C_{1}Cos(0)+C_{2}Sen(0)\\
    \Rightarrow C_{1} &= A
\end{align*}

Sustituyendo $C_{1}=A$ en $x(t)$:
\begin{align*}
    x(t) &= ACos(\omega_{0}t)+C_{2}Sen(\omega_{0}t)\\
    \Rightarrow \dot{x}(t) &= -A\omega_{0}Sen(\omega_{0}t) + C_{2}\omega_{0}Cos(\omega_{0}t)
\end{align*}

Aplicando la condición inicial $\dot{x}(t=0)=0$:
\begin{align*}
    0 &= -A\omega_{0}Sen(0) + C_{2}\omega_{0}Cos(0)\\
    0 &= C_{2}\omega_{0}\\
    \Rightarrow C_{2} &= 0
\end{align*}

Sustituyendo $C_{2}=0$ en $x(t)$:
\begin{equation*}
    x(t) = ACos(\omega_{0}t)
\end{equation*}

Se agrega una constante de desfase $\phi$ a la función $x(t)$ como desplazamiento dentro del ángulo del coseno. Esto con el fin de que la función sea compatible con cualquier sistema oscilatorio con un desfase inicial respecto al punto de equilibrio.
\begin{equation*}
    \therefore x(t) = ACos(\omega_{0}t-\phi)
\end{equation*}

Notemos que dadas las condiciones iniciales, en este caso particular $\phi=0$.\\
Si se hace uso de la identidad $Sen(x+\frac{\pi}{2}) = Cos(x)$ se puede reescribir $x(t)$ como:
\begin{align*}
    x(t) &= ASen((\omega_{0}t-\phi)+\frac{\pi}{2})\\
    x(t) &= ASen(\omega_{0}t-(\phi-\frac{\pi}{2}))\\
    \therefore x(t) &= ASen (\omega_{0}t-\delta)
\end{align*}
donde $\delta=\phi - \frac{\pi}{2}$