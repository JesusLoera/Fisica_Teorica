

 Solución.

 \begin{align*}
    \ddot{x} + \omega_{0}^{2} x &= 0 \\
    \frac{d^{2} x}{dt^{2}} + \omega_{0}^{2} x &= 0 
\end{align*}

Observación: $\frac{d^{2} x}{dt^{2}} + \omega_{0}^{2} x = 0 $ es una ecuación diferencial lineal
no homogénea de orden dos con coeficientes constantes.

\vspace*{0.3 cm}

Identificamos la ecuación característica: $ m^{2} + \omega_{0}^{2} = 0 $

Ahora buscamos despejar para m:

\begin{align*}
    m^{2} + \omega_{0}^{2} &= 0 \\
    m^{2}  &= - \omega_{0}^{2} \\
    \Rightarrow m &= 0 \pm \omega_{0} i
\end{align*}

Observación: Hemos obtenido raíces complejas conjugadas.

\vspace*{0.3 cm}

Identificamos $ a=0 y b=\omega $, la solución general de la ecuación diferencial
es de la forma:

\begin{align*}
    x(t) &= C_{1}e^{at}Cos(bt) + C_{2}e^{at}Sen(bt) \\
\end{align*}

Sustituyendo a y b:

\begin{align*}
    x(t) &= C_{1}e^{(0)t}Cos((\omega_{0})t) + C_{2}e^{(0)t}Sen((\omega_{0})t) \\
    x(t) &= C_{1}(1)Cos(\omega_{0}t) + C_{2}(1)Sen(\omega_{0}t) \\
    \Rightarrow x(t) &= C_{1}Cos(\omega_{0}t) + C_{2}Sen(\omega_{0}t)
\end{align*}

