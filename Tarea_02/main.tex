\documentclass[12pt]{article}
\usepackage{blindtext}
\usepackage[T1]{fontenc}
\usepackage[utf8]{inputenc}
\usepackage{amsmath}
\usepackage{amsfonts} 
\usepackage{color}
\usepackage{graphicx}
\usepackage{vmargin}
\usepackage{amssymb}
\usepackage[spanish]{babel}


\renewcommand{\abstractname}{Descripción}   % Cambiamos el nombre del abstract
\renewcommand{\contentsname}{Contenidos}   % Cambiamos el nombre de contents

\input{preambulo.tex}

%%%%%%%%%%%%%%%%%%%%%%%%%%%%%%%%%%%%%%%%%%%%%%%%%%%%%%%%%%%%%%%%%%%%%%%%%%%%
%%%%%%%%%%%%%%%%%%%%%%%%%%%%%%%%%%%%%%%%%%%%%%%%%%%%%%%%%%%%%%%%%%%%%%%%%%%%

\begin{document}

\begin{titlepage}
    \centering
    {\includegraphics[width=0.2\textwidth]{FCFM.png}\par}
    \vspace{1cm}
    {\bfseries\LARGE Universidad Autonoma de Nuevo León \par}
    \vspace{1cm}
    {\scshape\Large Facultad de Ciencias Físico Matemáticas \par}
    \vspace{3cm}
    {\scshape\Huge Proyecto 2  \par}
    \vspace{3cm}
    {\Large Autores: \par}
    \vfill
    {\Large Jesús Eduardo Loera Casas 1898887 \par}
    \vfill{\Large Cesar Efrén Valladares Rocha 1841555 \par}
    \vfill
    {\Large Vrani Chavez Islas 1990044 \par}
    \vfill
    {\Large \today \par}
\end{titlepage}

\tableofcontents			% Índice

\begin{center}
	\rule[0mm]{150mm}{0.1mm}		% Para dibujar una linea horizontal de
									% [elevación]{longitud}{grosor}
	\end{center}
	
	
\begin{abstract}		% ABSTRACT
  
\noindent 				% Anula la sangria de este parrafo
En este documento nuestro equipo presenta el Proyecto 1 del curso de mecánica teórica, 
donde planteamos la solución a los cuatro problemas problemas propuestos en el mismo.
\end{abstract}
	
\begin{center}
	\rule[0mm]{150mm}{0.1mm}
	\end{center}

\section{Problema del oscilador armónico}	
    
Encontrar las ecuaciones de movimiento y velocidad respecto al tiempo 
del siguiente oscilador armónico:

\vspace*{0.3 cm}

\begin{center}
    \begin{circuitikz}[american voltages]
        \draw (0,0) to [C=$C$] (4,0) -- (4,2) to [L, l=$L$] (0,2) -- (0,0);
    \end{circuitikz}
\end{center}

Por la ley de Kirchhoff:

\begin{align*}
    L \frac{dI}{dt} + \frac{1}{C} \int Idt = 0
\end{align*}

Donde $ I(t)=\dot{q} $        
                            
\section{Solución}
    

 Solución.

 \begin{align*}
    \ddot{x} + \omega_{0}^{2} x &= 0 \\
    \frac{d^{2} x}{dt^{2}} + \omega_{0}^{2} x &= 0 
\end{align*}

Observación: $\frac{d^{2} x}{dt^{2}} + \omega_{0}^{2} x = 0 $ es una ecuación diferencial lineal
homogénea de orden dos con coeficientes constantes.

\vspace*{0.3 cm}

Identificamos la ecuación característica: $ m^{2} + \omega_{0}^{2} = 0 $

Ahora buscamos despejar para m:

\begin{align*}
    m^{2} + \omega_{0}^{2} &= 0 \\
    m^{2}  &= - \omega_{0}^{2} \\
    \Rightarrow m &= 0 \pm \omega_{0} i
\end{align*}

Observación: Hemos obtenido raíces complejas conjugadas.

\vspace*{0.3 cm}

Identificamos $ a=0$ y $b=\omega_{0} $, la solución general de la ecuación diferencial
es de la forma:

\begin{align*}
    x(t) &= C_{1}e^{at}Cos(bt) + C_{2}e^{at}Sen(bt) \\
\end{align*}

Sustituyendo a y b:

\begin{align*}
    x(t) &= C_{1}e^{(0)t}Cos((\omega_{0})t) + C_{2}e^{(0)t}Sen((\omega_{0})t) \\
    x(t) &= C_{1}(1)Cos(\omega_{0}t) + C_{2}(1)Sen(\omega_{0}t) \\
    \Rightarrow x(t) &= C_{1}Cos(\omega_{0}t) + C_{2}Sen(\omega_{0}t)
\end{align*}



\section{Ecuaciones de movimiento}
    
Por lo tanto la ecuación del movimiento del oscilador armónico simple es:

\begin{itemize}
    \item $x(t) = C_{1}Cos(\omega_{0}t) + C_{2}Sen(\omega_{0}t)$
\end{itemize}

Que puede reescribirse como:

\begin{itemize}
    \item $x(t) = ACos(\omega_{0}t - \phi)$
    \item $x(t) = ASen(\omega_{0}t - \delta)$
\end{itemize}

\end{document}